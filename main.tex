\documentclass{article}
\usepackage[utf8]{inputenc}

\title{The Landscape of Big Data \& Gender: \\An Update}
\author{Bapu Vaitla}
\date{October 2020}

\begin{document}

\maketitle

This document reviews the latest advances at the intersection of big data, gender, and development, focusing on ongoing and published research from 2018 through 2020.

\textbf{Indicators of special interest:}
\begin{itemize}
    \item {\textbf{Monitoring the pandemic: primary health effects by sex.} COVID-19 infection and death rates; COVID-19 testing rates; COVID infection and death rates for health workers.} 
    \item{\textbf{Capacity to monitor the pandemic.} Cause of death, by communicable disease, ages 15-59 by sex; Proportion of children under 5 years of age who have been registered with a civil authority by sex.}
    \item{\textbf{Economic wellbeing: secondary effects from economic recession.} Employed population below international poverty line by sex; Employment distribution by economic activity by sex; Informal employment by sex; Prevalence of moderate or severe food insecurity in the adult population by sex; Proportion of time spent on unpaid domestic chores and care work by sex.}
     \item{\textbf{Human capital: secondary effects from containment measures.} Completion rate, lower secondary education by sex; Completion rate, primary education by sex; Completion rate, upper secondary education by sex; Enrolment in primary education by sex; Enrolment in secondary education by sex; Graduates from tertiary education by sex.}
    \item{\textbf{Women's health: secondary health effects on women and girls.} Adolescent birth rate (per 1,000 women aged 15-19 years); Height-for-age <-2 SD (stunting) by sex; Maternal mortality ratio• Prevalence of anemia among pregnant women; Prevalence of anxiety disorders (\%) by sex; Proportion of ever-partnered women and girls subjected to physical and/or sexual violence by a current or former intimate partner in the previous 12 months; Proportion of mothers who had postnatal contact with a health provider within 2 days of delivery; Proportion of women of reproductive age (aged 15-49 years) who have their need for family planning satisfied with modern methods; Weight-for-height <-2 SD (wasting) by sex.}
    \item{\textbf{Access to safety nets: mitigation.} Proportion of individuals who own a mobile telephone by sex; Proportion of population covered by social protection floors/systems by sex; Proportion of population with personal IDs by sex.}

\end{itemize}


\section{A review of the published literature}

Web of Science:
TS=("big data" or "digital data")  AND TS = ("gender" or "sex" or "women" or "woman" or "girls"): 731 results.
Abstracts scanned, 85 for further review.

Snowball: +2



Indicator search, WoS, no time restriction:

TS=("big data" or "digital data")  AND TS = ("gender" or "sex" or "women" or "woman" or "girls") AND TS = ("cause of death" OR "civil registration" OR "birth registration" OR "employment" OR "food insecurity" OR "food security" OR "domestic work" OR "care work" OR "unpaid work" OR "informal economy" OR  "primary education" OR "primary school"  or "primary enrollment" OR "primary enrolment" OR "secondary school" OR "secondary education" OR "secondary enrollment" OR "secondary enrolment" OR "completion rate" OR "tertiary education" OR "tertiary enrollment" OR "tertiary enrolment" OR "adolescent birth" OR "height-for-age" OR "stunting" OR "maternal mortality" OR "anemia" OR "anxiety" OR "physical violence" OR "sexual violence" OR "gender-based violence" OR "postnatal" OR "reproductive" OR "family planning" OR "weight-for-height" OR "wasting" OR "mobile phone" OR "cell phone" OR "social protection" OR "personal identification" OR "personal ID" OR "ID card" OR "identification card")
64 results, 11 retained


\subsection{Human-sourced data}

social media, blogs, vlogs, internet forums, wikis, internet searches, email/sms

\subsubsection{Peer-reviewed literature}


\textit{Social media}


Rizwan 2018: Check-ins on Weibo give sense of spatial density, temporal patterns, and geographical access 

Ullah et al 2019: Urban mobility and resource use--social media check-ins by gender. Some clue about access to spaces, health

Brandtzaeg 2017: FB doesn't equalize gender civic engagement, but replicates and reinforces offline patterns.

Martin 2020: natural disaster movement by gender using geotagged tweets--representative sample, including of younger people, who are often not represented well in surveys

Chenou 2019, niunamenos: active solicitation of user-generated content to fill gaps in official statistics, particularly on topics ignored by prevailing power.

Chen 2020: web crawling social media databases to explore particular gendered attitudes and behaviors---in this case, procrastination

Yuan 2018: Weibo checkin behavior possible instrument for assessing gender inequality across space.

Nebeker 2020: social media based health research growing; used to deliver interventions, recruit participants, surveillance, and feasibility in comparison to other research methods

Gonzalez 2019: digital protest networks reflect the structure of collective action and women's representation in that action


\textit{Email}
Widerstedt 2018: email content to assess gender bias in business counseling

\subsubsection{Preprints \& conference proceedings}

Flesch 2017: Official party membership rolls don't reflect active gender splits on Facebook walls, and connections between parties/ideologies. Possible mapping of norms, information sources, positions.



\subsubsection{Grey literature}

Pulse Lab Jakarta/JDS 2020: Combo of FB population density with admin data to look at covid transmission potential 

UN Global Pulse 2019 After Dark: Note the combination of qual info to inform big data topics (eg mobility)

\textit{Social media}

UNGP 2013 advo monitoring: tweets reflect engagement with women's health


\subsection{Process-mediated data}

health records, mobile phone data, credit card data, public transport usage, job application records, chips identfication, e-govt



\subsubsection{Peer-reviewed literature}

Macedonia et al 2017: big data already revolutionizing inference in biomedical sciences; digitized health records a potential enormous source of information about population patterns in reproductive and maternal health   

Heerwig 2018: political contribution data, by gender. Digitized records mandated to be open access by law can uncover patterns of voice and engagement

Croda 2018: Ministry of Health records collating injury data from hospitals, clinics, etc. Gender tags.


Suryavanshi 2020: mhealth apps collecting data on specialized topics, eg mother to child PMTCT


\subsubsection{Preprints \& conference proceedings}

Zhao 2018: cell phone dataset with gendered shopping patterns in Shanghai

Dourado 2017: govt programs generate large datasets to be used for evaluation and well-being monitoring


\subsubsection{Grey literature}

UNGP 2015 senegal migration: Seasonal mobility patterns using cell phone data, possible humanitarian early warning

Prahara 2019: Mobile networks and population displacement

UNGP 2017 Vanuatu: correlation between mobile phone data and socioeconomic status, especially education

\subsection{Media-sourced data}

tv/radio broadcast, podcast, digital newspapers

\subsubsection{Peer-reviewed literature}
\subsubsection{Preprints \& conference proceedings}

\subsubsection{Grey literature}

Africa's voices Kenya COVID 2020: SMS communications, backed by PSAs and interactive radio shows, analyzed with machine learning and human enables large number of interactions (Also Africa's Voices 2016 ebola and AVF 2020 kenya upper tana)

Africa's voices WTS 2018: sms, social media discussions on contraception; automated textual data analysis (NLP) of 100k messages/month---essentially social norms tracking

Pulse Lab Kampala 2019 automated speech recognition to transcribe radio conversations on violence against women; could track social norms this way. Note digitization of data flowing from traditional technologies...data is actually everywhere, big data is simply our rudimentary attempt to capture a small portion of it in digital, interpretable form. In that sense, our technologies are extraction tools, not generators, of data.

\subsection{Crowdsourced data}
citizen-generated, images collection, volunteered geographical information

\subsubsection{Peer-reviewed literature}
\subsubsection{Preprints \& conference proceedings}

\subsubsection{Grey literature}



\subsection{Machine-generated data}

raod sensor data, smart meter electricity, scanners, satellite/aerial imagery, traffic loops webcams, vessel id, internet of things

\subsubsection{Peer-reviewed literature}
\subsubsection{Preprints \& conference proceedings}

\subsubsection{Grey literature}



\section{Gender prediction}

Yang 2016: 97\% accuracy with a combination of query based semantic, visual, and name methods, in a voting framework.

Lopez-Santamaria et al 2019: For some types of media (here Pinterest), remains very difficult (little improvement over classifying all as women)

Duong et al. 2016: Online shopping patterns, feature selection improves slightly on textual analysis

\section{Interviews}


Interview list

\begin{itemize}
    \item
\end{itemize}

Other docs to review:

Reed, P.J.; Khan, M.R.; Blumenstock, J. Observing gender dynamics and disparities with mobile phone metadata. In Proceedings of the Eighth International Conference on Information and Communication Technologies and Development, Ann Arbor, MI, USA, 3–6 June 2016.

Stefanone, M.A.; Huang, Y.C.; Lackaff, D. Negotiating Social Belonging: Online, Offline, and In-Between. In Proceedings of the 44th Hawaii International Conference on System Science, Kauai, HI, USA, 4–7 January 2011; pp. 1–10.

Blumenstock, J.E.; Gillick, D.; Eagle, N. Who’s calling? Demographics of mobile phone use in Rwanda. Transportation 2010, 32, 2–5.

Rizwan, M.; Wanggen, W.; Cervantes, O.; Gwiazdzinski, L. Using Location-Based Social Media Data to Observe Check-In Behavior and Gender Difference: Bringing Weibo Data into Play. ISPRS Int. J. Geo-Inf. 2018, 7, 196

FROM YANG AND YUAN 2016

C. Tang, K. Ross, N. Saxena, and R. Chen, “Whats in a name: a study of names, gender inference, and gender behavior in facebook,” in Database Systems for Adanced Applications. Springer, 2011, pp. 344–356.

[4] J. D. Burger, J. Henderson, G. Kim, and G. Zarrella, “Discriminating gender on twitter,” in Proceedings of the Conference on Empirical Methods in Natural Language Processing. Association for Computational Linguistics, 2011, pp. 1301–1309. [5] C. Peersman, W. Daelemans, and L. Van Vaerenbergh, “Predicting age and gender in online social networks,” in Proceedings of the 3rd international workshop on Search and mining user-generated contents. ACM, 2011, pp. 37–44. [6] A. Kokkos and T. Tzouramanis, “A robust gender inference model for online social networks and its application to linkedin and twitter,” First Monday, vol. 19, no. 9, 2014.

A. Mukherjee and B. Liu, “Improving gender classification of blog authors,” in Proceedings of the 2010 conference on Empirical Methods in natural Language Processing. Association for Computational Linguistics, 2010, pp. 207–217. [9] C. Fink, J. Kopecky, and M. Morawski, “Inferring gender from the content of tweets: A region specific example.” in ICWSM, 2012.

F. Al Zamal, W. Liu, and D. Ruths, “Homophily and latent attribute inference: Inferring latent attributes of twitter users from neighbors.” ICWSM, vol. 270, 2012.

J. S. Alowibdi, U. A. Buy, and P. Yu, “Language independent gender classification on twitter,” in Advances in Social Networks Analysis and Mining (ASONAM), 2013 IEEE/ACM International Conference on. IEEE, 2013, pp. 739–743. [12] J. S. Alowibdi, U. A. Buy, and S. Y. Philip, “Say it with colors: Language-independent gender classification on twitter,” in Online Social Media Analysis and Visualization. Springer, 2014, pp. 47–62.

X. Ma, Y. Tsuboshita, and N. Kato, “Gender estimation for sns user profiling using automatic image annotation,” in Multimedia and Expo Workshops (ICMEW), 2014 IEEE International Conference on. IEEE, 2014, pp. 1–6.

W. Liu and D. Ruths, “What’s in a name? using first names as features for gender inference in twitter.” in AAAI Spring Symposium: Analyzing Microtext, vol. 13, 2013, p. 01.

---

Peersman, C., Daelemans, W., & Van Vaerenbergh, L. (2011, October). Predicting age and gender in online social networks. In Proceedings of the 3rd international workshop on Search and mining user-generated contents (pp. 37-44). ACM.

Levi, G., & Hassner, T. (2015). Age and gender classification using convolutional neural networks. In Proceedings of the IEEE Conference on Computer Vision and Pattern Recognition Workshops (pp. 34-42).

Agrawal, M., & Gonalves, T. (2016). Age and Gender Identification using Stacking for Classification Notebook for PAN at CLEF 2016.

Schwartz, H. A., Eichstaedt, J. C., Kern, M. L., Dziurzynski, L., Ramones, S. M., Agrawal, M., ... & Ungar, L. H. (2013). Personality, gender, and age in the language of social media: The open-vocabulary approach. PloS one, 8(9), e73791.

Goswami, S., Sarkar, S., & Rustagi, M. (2009, March). Stylometric analysis of bloggers age and gender. In Third International AAAI Conference on Weblogs and Social Media.

Rangel, F., Rosso, P., Potthast, M., & Stein, B. (2017). Overview of the 5th author profiling task at pan 2017: Gender and language variety identification in twitter. Working Notes Papers of the CLEF.

T. M. Phuong, and D. V. Phuong, “Gender prediction using browsing history,” Proceedings of the Fifth International Conference KSE 2013, Volume 1. pp. 271-283, 2013.

F. Rangel, and P. Rosso, “Use of language and author profiling: Identification of gender and age,” In Natural Language Processing and Cognitive Science, p. 177, 2013.

\end{document}
