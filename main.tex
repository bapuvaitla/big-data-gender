\documentclass{article}
\usepackage[utf8]{inputenc}

\title{The Landscape of Big Data \& Gender: \\An Update}
\author{Bapu Vaitla}
\date{October 2020}

\begin{document}

\maketitle

This document reviews the latest advances at the intersection of big data, gender, and development, focusing on ongoing and published research from 2018 through 2020.

\textbf{Indicators of special interest:}
\begin{itemize}
    \item {\textbf{Monitoring the pandemic: primary health effects by sex.} COVID-19 infection and death rates; COVID-19 testing rates; COVID infection and death rates for health workers.} 
    \item{\textbf{Capacity to monitor the pandemic.} Cause of death, by communicable disease, ages 15-59 by sex; Proportion of children under 5 years of age who have been registered with a civil authority by sex.}
    \item{\textbf{Economic wellbeing: secondary effects from economic recession.} Employed population below international poverty line by sex; Employment distribution by economic activity by sex; Informal employment by sex; Prevalence of moderate or severe food insecurity in the adult population by sex; Proportion of time spent on unpaid domestic chores and care work by sex.}
     \item{\textbf{Human capital: secondary effects from containment measures.} Completion rate, lower secondary education by sex; Completion rate, primary education by sex; Completion rate, upper secondary education by sex; Enrolment in primary education by sex; Enrolment in secondary education by sex; Graduates from tertiary education by sex.}
    \item{\textbf{Women's health: secondary health effects on women and girls.} Adolescent birth rate (per 1,000 women aged 15-19 years); Height-for-age <-2 SD (stunting) by sex; Maternal mortality ratio• Prevalence of anemia among pregnant women; Prevalence of anxiety disorders (\%) by sex; Proportion of ever-partnered women and girls subjected to physical and/or sexual violence by a current or former intimate partner in the previous 12 months; Proportion of mothers who had postnatal contact with a health provider within 2 days of delivery; Proportion of women of reproductive age (aged 15-49 years) who have their need for family planning satisfied with modern methods; Weight-for-height <-2 SD (wasting) by sex.}
    \item{\textbf{Access to safety nets: mitigation.} Proportion of individuals who own a mobile telephone by sex; Proportion of population covered by social protection floors/systems by sex; Proportion of population with personal IDs by sex.}

\end{itemize}


\section{A review of the published literature}

Web of Science:
TS=("big data" or "digital data")  AND TS = ("gender" or "sex" or "women" or "woman" or "girls"): 731 results.
Abstracts scanned, 85 for further review.

Snowball: +2



Indicator search, WoS, no time restriction:

TS=("big data" or "digital data")  AND TS = ("gender" or "sex" or "women" or "woman" or "girls") AND TS = ("cause of death" OR "civil registration" OR "birth registration" OR "employment" OR "food insecurity" OR "food security" OR "domestic work" OR "care work" OR "unpaid work" OR "informal economy" OR  "primary education" OR "primary school"  or "primary enrollment" OR "primary enrolment" OR "secondary school" OR "secondary education" OR "secondary enrollment" OR "secondary enrolment" OR "completion rate" OR "tertiary education" OR "tertiary enrollment" OR "tertiary enrolment" OR "adolescent birth" OR "height-for-age" OR "stunting" OR "maternal mortality" OR "anemia" OR "anxiety" OR "physical violence" OR "sexual violence" OR "gender-based violence" OR "postnatal" OR "reproductive" OR "family planning" OR "weight-for-height" OR "wasting" OR "mobile phone" OR "cell phone" OR "social protection" OR "personal identification" OR "personal ID" OR "ID card" OR "identification card")
64 results, 11 retained


\subsection{Human-sourced data}

social media, blogs, vlogs, internet forums, wikis, internet searches, email/sms

Balicer 2018: Strengths and weaknesses for health surveillance

\subsubsection{Social media}

Ma 2020: Explicit recognition of social media as an emotions database, sentiment analysis. Gender differences, the interaction of green space and how motions are created, how gender inequality is influenced and potentially amplified or attenuated by the design of urban space/nature. 

Larrondo 2019: Twitter networks can give a sense of the degree or type of politicization (in this case, feminism), and how network structure can given a sense of the potential for change

Rizwan 2018a and 2018b, Khan 2020, Muhammad 2019: Check-ins on Weibo give sense of spatial density, temporal patterns, and geographical access 

Ullah et al 2019, 2020 (green space): Urban mobility and resource use--social media check-ins by gender. Some clue about access to spaces, health

Clark 2018: SC decision has effects on public opinion as measured by Twitter re gay marriage. Suggests social media's utility as a barometer of changing social norms, and hte dyanmics of those norms as they relate to discrete events

Friedman 2019: gendered word embeddings reflect societal gender gaps more generally; Twitter as a tool for monitoring changes in social norms as reflected in gendered language

Brandtzaeg 2017: FB doesn't equalize gender civic engagement, but replicates and reinforces offline patterns.

Kosinski 2013: personality traits, friend networks, demographics, values all predictable through FB likes--a kind of emotional monitoring system; the foundation is all there

Martin 2020: natural disaster movement by gender using geotagged tweets--representative sample, including of younger people, who are often not represented well in surveys

Chenou 2019, niunamenos: active solicitation of user-generated content to fill gaps in official statistics, particularly on topics ignored by prevailing power.

Olteanu 2018: hate speech following extremist events, could be checked for gendered events as well

Nambisan 2015: Not just the exact content of tweets but also the pattern (eg. repetition) and style (eg, ) can be used to infer depression

UNGP 2017 El Salvador: twitter to assess attitudes towards policy

Chen 2020: web crawling social media databases to explore particular gendered attitudes and behaviors---in this case, procrastination

Thomas 2019: Twitter to assess religiosity, social norms

Yuan 2018: Weibo checkin behavior possible instrument for assessing gender inequality across space.

Nebeker 2020: social media based health research growing; used to deliver interventions, recruit participants, surveillance, and feasibility in comparison to other research methods

UNGP 2015a, b vaccination: networks of idea spread/influence can be constructed, shedding light into the evolution of social norms; example here of vaccination discussions on FB and twitter. Nigeria, India (a), Indonesia (b)

Wang 2013: Facebook status updates are gendered, reflect some degree of cultural difference. Less gendered among teenagers, which might reflect narrower band of concern during teenage years, or amount of time spent among same peer group, etc.

Tomeny 2017: Tweets to check beliefs, in this case about vaccination, and break out demographics of belief systems. Could be used to trace the flow of ideas in a society

Gonzalez 2019: digital protest networks reflect the structure of collective action and women's representation in that action

UNFP 2011 unemployment: Sentiment anlaysis of online conversations on blogs, forums, etc can provie leading indicators of economic downturn; no gender component here, but potential

UNGP 2014 food prices: Twitter as a predictor of food price changes

UNGP 2014 contraception: analyzing attitudes on FB and SMS-based polling system

UNGP

Boy 2018: Twitter as a spce-time map of the the global spread of ideas and norms, in this case #hijabfashion

UNGP 2013 advo monitoring: tweets reflect engagement with women's health

UNGP 2014 indonesian tweets: even without representative samples, tweets can give leading indications of economic crises, especially in urban areas with large twitter presence; relative frequency is the key here

UNGP 2011: Twitter as surveillance system of crisis perceptions more generally; applicable cross-culturally, as each population of users builds their own twitter-specific forms of communication. Still biased sample; also danger of looking under the streetlight--twitter data is free

UN Global Pulse 2019 After Dark: Note the combination of qual info to inform big data topics (eg mobility)

Rizwan 2017: check-ins can reconstruct economic, social behavior, including gendered patterns of access to space, services, resources

UNGP 2017, uganda debates: civic engagement barometer. need to adjust to account for urban, high-income bias. Some sentiment analysis possible

UNGP 2017, refugee sentiment: tracking host and refugee reactions, including instances of xenophobia; twitter representativeness problem. [what is the essence of twitter? a thought megaphone; still have to filter, probably power law distribution of audience]

UNGP 2014 migration: Google searches and migration 

\subsubsection{Other}

Gallus 2020: Wikipedia conversations, insight into gendered differences in communication, informational content. At least a partial reflection of social norms. Also insight into how institutions affect gendered communication styles, though not sure if 'certain' women rise or institutional norms force change at higher levels of hierarchical systems

Priest 2016: Health Q&A site gives insights into gendered health concerns--a picture not only of social norms, what's known and what's not, but also the state of health education; presumably change over time

Wang 2016: google search correlation with incidence of dementia is gendered, possibly because of differences in health-seeking behavior, linked to social norms

Widerstedt 2018: email content to assess gender bias in business counseling

Cisco 2019: gendered differences in gofundme behavior, testing hypothesis about social norms

Flesch 2017: Official party membership rolls don't reflect active gender splits on Facebook walls, and connections between parties/ideologies. Possible mapping of norms, information sources, positions.

Pulse Lab Jakarta/JDS 2020: Combo of FB population density with admin data to look at covid transmission potential 

Thelwall 2018: Algos need to be gender-corrected if they are to accurately analyze/classify sentiment [using TripAdvisor reviews from UK]

Jimenez 2020: Google search trends disaggregated by gender; correlations between search terms and suicide is gendered, reflecting differences in communication styles; if these can be controlled for, possible prediction of suicide risk

Strathdee 2019: Many sites available for monitoring knowledge, needs, and behavior around HIV/AIDS -- disaggregation can be used to estimate prevalence for highly disaggregated voters

Kalimari 2019: personality traits, belief systems, values, gender all predictable to an extent by web browsing history

Bergstrom 2018: dating site information to assess reasons for dynamics between couples, in this case age differences [perhaps more general investigation of dating site data]

UNGP 2017: spatial distress signals to track rescues in Mediterranean

\subsection{Process-mediated data}

health records, mobile phone data, credit card data, public transport usage, job application records, chips identification, e-govt



\subsubsection{Peer-reviewed literature}

Karakurt 2017: explorys database, even with deidentification, allows quantification of the morbidity burden assocaited with IPV, as well as possible prevalence--certainly relative prevalence over time, which may be especially important during recessions and other shocks.

Macedonia et al 2017: big data already revolutionizing inference in biomedical sciences; digitized health records a potential enormous source of information about population patterns in reproductive and maternal health   

Heerwig 2018: political contribution data, by gender. Digitized records mandated to be open access by law can uncover patterns of voice and engagement

Kashyap et al 2020, Fatehkia 2018: social media, internet searches give rise to metadata, e.g., on gender and number of users. Can be used to quantify the global digital gender divide, itself a predictor of how useful/unbiased various forms of big data are. A meta-signal of gendered big data

Balicer 2018: Strengths and weaknesses of health records [overall look at virtual and real digital trails as well, i.e., social media and transactional data]. Strengths: Enormous, comprehensive, longitudinal; Weaknesses: poor standardization, leaves out those without access to health care

Croda 2018: Ministry of Health records collating injury data from hospitals, clinics, etc. Gender tags.

Masso 2019: CDRs giving insight into mobility patterns by gender in Estonia, *age, etc. May yield insight into gendered access to resources in other countries, leisure time, etc.

Rhoads 2020: CDR data can illuminate patterns of segregation and inter-group communication

Suryavanshi 2020: mhealth apps collecting data on specialized topics, eg mother to child PMTCT

Dontheneni 2019: health records used to estimate gendered etc. patterns of disease prevalence; needs coordination across providers, especially in anonymization

Vermund 2020: Health records to assess HIV outcomes risk

Matter 2015: Existing political databases, query tool to facilitate gender analysis

Hristova 2016: Postal flows correlate to socioeconomic indicators, but (in this paper) lack a gender component; still, illustrates the rise of digital data for behaviors that were not original digital, increasingly recorded as being so

Chandir 2018: predictive analytics to identify those at greatest risk, even if data from a particular locale isn't present--another form of modeling a continuous landscape, in this case of risk of defaulting in vaccination

Cigsar 2018: gender-disaggregated credit default risk

Davies 2018: retailer loyalty card data examines self-medication patterns

Song 2018: E-commerce data as a novel test of evolutionary hypotheses

\subsubsection{Preprints \& conference proceedings}

Reed 2016: Cell phone metadata can be linked to indicators of gender parity, e.g., school enrollment. More dynamic picture of enrollment, plus perhaps effects post-crisis

Zhao 2018: cell phone dataset with gendered shopping patterns in Shanghai

Dourado 2017: govt programs generate large datasets to be used for evaluation and well-being monitoring


\subsubsection{Grey literature}

UNGP 2015 senegal migration: Seasonal mobility patterns using cell phone data, possible humanitarian early warning

UNGP 2016: point of sale transactions, atm withdrawals to understand post-disaster response

Prahara 2019: Mobile networks and population displacement

UNGP 2014: The landscape of disaster communications using CDRs

UNGP 2017 Vanuatu: correlation between mobile phone data and socioeconomic status, especially education

UNGP 2018 financial: bank account information is ubiquitous, increasingly covering all classes. Gender is almost always collected; can illuminate bottlenecks to women's use of financial services

UNGP 2017 mobile money: m-money as a signal of financial access, economic well-being

UNGP 2017 debit cards: with demographic info,  mobility signatures in debit cards used as indicators of spatial differences in wealth within cities

	\subsection{Media-sourced data}

tv/radio broadcast, podcast, digital newspapers

\subsubsection{Peer-reviewed literature}


Lee 2019: article commenters reflect current state of gendered social norms--which topics receive engagement. Opens the door to questions about what drives differences in gender engagement

\subsubsection{Preprints \& conference proceedings}

Huluba et al: Audit of the entire .uk domain to ascertain women's representation in employment roles and sectors

\subsubsection{Grey literature}

Africa's voices Kenya COVID 2020: SMS communications, backed by PSAs and interactive radio shows, analyzed with machine learning and human enables large number of interactions (Also Africa's Voices 2016 ebola and AVF 2020 kenya upper tana)

Africa's voices WTS 2018: sms, social media discussions on contraception; automated textual data analysis (NLP) of 100k messages/month---essentially social norms tracking

Pulse Lab Kampala 2016: Automated speech to text tool, geographical and time tags, identify substance of conversations in Uganda occurring over radio (7.5m words/day)

Pulse Lab Kampala 2019 automated speech recognition to transcribe radio conversations on violence against women; could track social norms this way. Note digitization of data flowing from traditional technologies...data is actually everywhere, big data is simply our rudimentary attempt to capture a small portion of it in digital, interpretable form. In that sense, our technologies are extraction tools, not generators, of data.

Also UNGP 2019a, b on talk shows conversations about health

UNGP 2019, somalia displacement: predictors of refugee movement

PulseWeb 2011: topic content and relationships, evolution of information/media visibility over time; potentially applicable to gendered issues.

\subsection{Crowdsourced data}
citizen-generated, images collection, volunteered geographical information

\subsubsection{Peer-reviewed literature}
\subsubsection{Preprints \& conference proceedings}

\subsubsection{Grey literature}



\subsection{Machine-generated data}

[virtual digital trails]

road sensor data, smart meter electricity, scanners, satellite/aerial imagery, traffic loops webcams, vessel id, internet of things

Balicer 2018: Strengths and weaknesses for health surveillance


\subsubsection{Peer-reviewed literature}

Moreira 2019: Internet of Things devices can allow predictive diagnosis of postpartum depression 

Mohan 2019: data for health evaluations flows directly from apps to database

Nguyen 2020: built environment predictors of COVID; something similar could be used to assess gendered outcomes


\subsubsection{Preprints \& conference proceedings}

\subsubsection{Grey literature}

UNGP 2019 indonesia financial access points: spatial mapping of access, but lacks gender component

UNGP 2014 machine roof counting: interesting proxy for income, but no gender component

\section{Gender prediction}

Kosinski 2013: FB likes can predict gender >90\% accuracy

Yang 2016: 97\% accuracy with a combination of query based semantic, visual, and name methods, in a voting framework.

Lopez-Santamaria et al 2019: For some types of media (here Pinterest), remains very difficult (little improvement over classifying all as women)

Duong et al. 2016: Online shopping patterns, feature selection improves slightly on textual analysis

Gao et al. 2019: classification algo development continues

Gu 2018: redundancy in web pages can help predict gender accurately; speaks to the more general point of triangulation and integration of data sources

Jia 2016: facial recognition accuracy reaching 99\% in weakly labeled datasets

Manik 2019: combination image and name classification accuracy reaches 99\% in twitter datasets

Radford 2017: reinserting gender theory (institutional, interactional, individual) can help improve gender prediction algo performance

Sangaralingam 2018: many smartphone activity, environment, and engagement based features available to predict gender

\section{Interviews}


Interview list

\begin{itemize}
    \item
\end{itemize}


\section{Conclusions}

Big problem with many digital datasets is of course sampling frame bias (Hargittai 2015)

2x2 cell: sampling frames (possibly 3x3 with ease of data access; also consider age; perhaps privacy risk)
Class-widespread, gender-missing/limited: road sensors, satellite imagery, electricity, public transport usage
Class-widespread, gender-widespread: cell phone data (but gender problem), health records (coordination), crowd-sourced data, emails (but access problem)
Class-missing/limited, gender-missing/limited: Twitter (but ease of data access), Weibo, job applications, wikis?
Class-missing/limited, gender-widespread: Facebook (but difficult access), internet searches, Transactional data, IoT, credit card data

The ability of big data to close the gender data gap depends on first closing the digital gender divide (Montiel 2018), especially since much of the sophisticated data is coming from user-generated content.


\textit{Privacy}

Better understanding of aggregation vs utility trade-offs needed; often contextual--but who has access and when, specifically? (UNGP 2015)

Uncovering social relationships (di Luzio 2018)


Other docs to review:

Reed, P.J.; Khan, M.R.; Blumenstock, J. Observing gender dynamics and disparities with mobile phone metadata. In Proceedings of the Eighth International Conference on Information and Communication Technologies and Development, Ann Arbor, MI, USA, 3–6 June 2016.

Stefanone, M.A.; Huang, Y.C.; Lackaff, D. Negotiating Social Belonging: Online, Offline, and In-Between. In Proceedings of the 44th Hawaii International Conference on System Science, Kauai, HI, USA, 4–7 January 2011; pp. 1–10.

Blumenstock, J.E.; Gillick, D.; Eagle, N. Who’s calling? Demographics of mobile phone use in Rwanda. Transportation 2010, 32, 2–5.

Rizwan, M.; Wanggen, W.; Cervantes, O.; Gwiazdzinski, L. Using Location-Based Social Media Data to Observe Check-In Behavior and Gender Difference: Bringing Weibo Data into Play. ISPRS Int. J. Geo-Inf. 2018, 7, 196

FROM YANG AND YUAN 2016

C. Tang, K. Ross, N. Saxena, and R. Chen, “Whats in a name: a study of names, gender inference, and gender behavior in facebook,” in Database Systems for Adanced Applications. Springer, 2011, pp. 344–356.

[4] J. D. Burger, J. Henderson, G. Kim, and G. Zarrella, “Discriminating gender on twitter,” in Proceedings of the Conference on Empirical Methods in Natural Language Processing. Association for Computational Linguistics, 2011, pp. 1301–1309. [5] C. Peersman, W. Daelemans, and L. Van Vaerenbergh, “Predicting age and gender in online social networks,” in Proceedings of the 3rd international workshop on Search and mining user-generated contents. ACM, 2011, pp. 37–44. [6] A. Kokkos and T. Tzouramanis, “A robust gender inference model for online social networks and its application to linkedin and twitter,” First Monday, vol. 19, no. 9, 2014.

A. Mukherjee and B. Liu, “Improving gender classification of blog authors,” in Proceedings of the 2010 conference on Empirical Methods in natural Language Processing. Association for Computational Linguistics, 2010, pp. 207–217. [9] C. Fink, J. Kopecky, and M. Morawski, “Inferring gender from the content of tweets: A region specific example.” in ICWSM, 2012.

F. Al Zamal, W. Liu, and D. Ruths, “Homophily and latent attribute inference: Inferring latent attributes of twitter users from neighbors.” ICWSM, vol. 270, 2012.

J. S. Alowibdi, U. A. Buy, and P. Yu, “Language independent gender classification on twitter,” in Advances in Social Networks Analysis and Mining (ASONAM), 2013 IEEE/ACM International Conference on. IEEE, 2013, pp. 739–743. [12] J. S. Alowibdi, U. A. Buy, and S. Y. Philip, “Say it with colors: Language-independent gender classification on twitter,” in Online Social Media Analysis and Visualization. Springer, 2014, pp. 47–62.

X. Ma, Y. Tsuboshita, and N. Kato, “Gender estimation for sns user profiling using automatic image annotation,” in Multimedia and Expo Workshops (ICMEW), 2014 IEEE International Conference on. IEEE, 2014, pp. 1–6.

W. Liu and D. Ruths, “What’s in a name? using first names as features for gender inference in twitter.” in AAAI Spring Symposium: Analyzing Microtext, vol. 13, 2013, p. 01.

---

Peersman, C., Daelemans, W., & Van Vaerenbergh, L. (2011, October). Predicting age and gender in online social networks. In Proceedings of the 3rd international workshop on Search and mining user-generated contents (pp. 37-44). ACM.

Levi, G., & Hassner, T. (2015). Age and gender classification using convolutional neural networks. In Proceedings of the IEEE Conference on Computer Vision and Pattern Recognition Workshops (pp. 34-42).

Agrawal, M., & Gonalves, T. (2016). Age and Gender Identification using Stacking for Classification Notebook for PAN at CLEF 2016.

Schwartz, H. A., Eichstaedt, J. C., Kern, M. L., Dziurzynski, L., Ramones, S. M., Agrawal, M., ... & Ungar, L. H. (2013). Personality, gender, and age in the language of social media: The open-vocabulary approach. PloS one, 8(9), e73791.

Goswami, S., Sarkar, S., & Rustagi, M. (2009, March). Stylometric analysis of bloggers age and gender. In Third International AAAI Conference on Weblogs and Social Media.

Rangel, F., Rosso, P., Potthast, M., & Stein, B. (2017). Overview of the 5th author profiling task at pan 2017: Gender and language variety identification in twitter. Working Notes Papers of the CLEF.

T. M. Phuong, and D. V. Phuong, “Gender prediction using browsing history,” Proceedings of the Fifth International Conference KSE 2013, Volume 1. pp. 271-283, 2013.

F. Rangel, and P. Rosso, “Use of language and author profiling: Identification of gender and age,” In Natural Language Processing and Cognitive Science, p. 177, 2013.

\end{document}
